\documentclass[12pt, letterpaper]{article}

\usepackage{hyperref}
\hypersetup{colorlinks=true, urlcolor=blue, citecolor=magenta}

\title{GrandScan: A Portable, Scalable Laser Scanner}
\author{Noah Halford\\
    \href{mailto:nhalford@uchicago.edu}{nhalford@uchicago.edu}
        \and
        Catherine Moresco\\
        \href{mailto:cmoresco@uchicago.edu}{cmoresco@uchicago.edu}}
        \date{{\Large Team Scanner Planner}\\[5mm] June 3, 2016}

\begin{document}

\maketitle

% 1/2 page
% Describe what you planned to accomplish with the project.
% What specifically will be created, demonstrated, invented?
% What do you expect to learn from this project?
\section{Specific Aims of the Project} \label{aims}

One possible use case for 3D printers is object replication. While it certainly is possible
to design 3D models of real-world objects in CAD software, there are a number of drawbacks
to this approach: it may require many iterations to get right, it may be time-consuming, and
it requires a very particular set of skills. One alternative is to use a 3D scanner, which
simplifies and automates the process of digitally fabricating the desired object.

The goal of this project was to design and build a 3D scanner that could ultimately be used
to produce STL file approximations of real-world objects. Of course, because we were doing
this cheaply, we did not anticipate results that might resemble those of an industrial-quality
laser scanner: our attempt to create a laser scanner would be considered successful if we were
able to produce STL files that were recognizable as the objects they represented. We did not
expect that our resulting STLs would necessarily be completely faithful representations of the
original objects.

As will be discussed later, many of the existing and easily available designs for DIY
3D scanners are lacking in either portability or scalability. This led us
to attempt to create a laser scanner which would be both portable and scalable, as well
as affordable.

From this project, we hoped to learn both how laser scanning works and what some of its
limitations are.

% 1 page
% Discuss previous work on this problem by others.
% Discuss sources of information or designs you are leveraging.
% Discuss shortcomings of previous efforts.
\section{Background Research} \label{background}

Laser scanning is a technology that has already been very well-developed; going into
this project, we knew that it was unlikely for us to make significant improvements
over existing designs. Our goals were to design a laser scanner that would be cheap
and would work (as described in Section~\ref{aims}), but, perhaps most importantly,
to learn about how laser scanners work. Instructions for a number of cheap DIY laser
scanners already exist on the Internet, and we used these as inspiration for our
project---although we did design our own hardware and software.

The two main preexisting DIY scanners that we usd as inspiration for this project
were~\cite{sardau} and~\cite{dentroman}. 

\subsection{Approaches to design}
We found~\cite{sardau} first and based
our initial design off of this idea, because it was small and easily portable. However,
in building and trying to use a scanner derived from this design, we decided that it
was too small. In testing this design, we found ourselves wanting to scan larger objects;
ideally, we would like to be able to scan a human.

The scanner presented in~\cite{dentroman} is not scalable either, but it allows for the
scanning of somewhat larger objects. However, this scanner is much less portable and much
more unwieldy than~\cite{sardau}. We liked the concept of this design, but we were unhappy
with the design itself.

It appears that there are no affordable, scalable, and portable DIY laser scanners.
We sought to change this.

\subsection{Approaches to measurement and mathematics}
While designs vary wildly from laser scanner to laser scanner, the mathematical methods developed for the capturing of data and the construction of 3D models are fairly well-established; that being said, there are several different methods available, and it is interesting to note that each of these scanners take a different approach.

The Sarduscan takes the more canonical approach to laser scanning: that of triangulation. The laser is mounted a known angle to the line of sight of the camera; thus, distance from the camera to the target can be determined by the apparent location of the laser point in the camera's field of view. Laser locations are found by taking two images--one with the laser off and one with the laser on--and finding the difference between them. For a more detailed explanation of this, see ~\cite{triangulation}.

In contrast, the Dentroman786 scanner doesn't use this approach, instead using a mathematical approach based on deformations of light more akin to structured-light scanning seen in devices like the Kinect. The angle between the camera and the laser is constantly changing; to find the location of the points in relation to the camera, we find the lines where the laser intersects the wall and the floor, and use these to define a basis for a coordinate transform to the camera's frame.

We experimented with both of these techniques in different iterations of GrandScan.

% 2 pages
% Describe in some detail how you went about designing and building.
% Describe the steps you went through to develop ideas and prototypes.
% Describe any changes in your plan from the pre-proposal stage.
\section{Approach}

In our exploration of the different technologies and techniques used to create DIY laser scanners, our development cycle went through several iterations.

\subsection{GrandScan 1.0}
Initially, we modeled our laser scanner after
the one described in~\cite{sardau}. We designed and constructed a laser scanner consisting of  two webcams and tqo
lasers between them mounted on a circular arc, with a turntable that was mounted on a stepper motor
at the center of the arc. The target would sit on the turntable as it was scanned. At each step of the motor, each camera would capture
an image; we the planned to combine data about rotation with a number of distance
measurements (the distance between the cameras, between the cameras and the lasers, etc.)
in order to produce a 3D model representation of the object we wished to scan using triangulation, as the Sarduscan scanner does. Our prototype used
an Arduino attached to the stepper motor to control the turntable. A Python program
communicating with the Arduino allowed us rotate the stepper motor (that is, the turntable)
one step at a time, and take a photo from each webcam between steps. This produced
1024 images (512 per camera).
After building a protoype of this scanner, we were dissatisfied with it for several reasons.
Firstly, the configuration was not adjustable. The parts were 3D printed, and, after several tried, fit together nicely and looked quite nice; however, when actually scanning an object, we realized that we wanted to have the freedom to adjust the heights of the lasers and the angles of the cameras--something that our entirely 3D printed design did not allow.
Additionally, the webcams did not offer the resolution we were hoping for. Their autofocus function was ultimately unsatisfactory, and the images obtained as data were blurry and poorly resolved; even to the naked eye, structure of the target was hard to discern--we were not optimistic about the capabilities of software to accurately generate a point cloud.
Additionally, we realized that scanning on this level was--well--a little boring. After looking around for objects to test our scanner on, we realized that most objects that would fit on the print bed were rotationally symmetric, anyway (cups, salt shakers, bottles, etc); there were very few things that would fit on the 8-inch print bed that we would even want to scan. So, we decided to think a little bigger and begin a second iteration of our design.

\subsection{GrandScan 2.0}
Our second design was more heavily inspired by~\cite{dentroman}, using a similar laser-gantry-style scanning method and a similar mathematical approach. Ours, however, was much bigger and much simpler.

The design consists of three meter-long rods, joined at right angles at their ends by a 3D-printed joint. On the upright rod, there is a holder for an iPhone that can slide up and down to be of adjustable height. A gantry slides over one of the rods on the ground, on which is mounted the laser and the controlling Arduino. These three rods define a space of one cubic meter that represents the print volume; one "leg" is the track for the gantry and the width of the space, the other "leg" maintains a distance from the wall and provides support, and the third upright rod is a mount for the camera and represents the maximum printable height. Theoretically, a space of this size could even be used to scan a human--with eye protection, of course!

Mathematically, we also started from scratch, using the same technique as ~\cite{dentroman}. We first used a Hough transform to attempt to find the lines where the laser scans the floor and the wall; using these, we then transformed the other segments of the line using a coordinate transform to determine their location in our model. 

We decided to switch to use of an iPhone to record data. This gives us the benefit of higher resolution than our cheap webcams, but also comes at the cost of not being directly able to transfer images to the controlling program, or to control the camera from the program. While, eventually, it might be interesting to write an app to manage this sort of control, we unfortunately did not have the time to do so. Instead, we recorded photos and videos from that fixed position, and then uploaded them to the computer for processing.




% 1/2 page
% Describe any novel technology that you used or developed during the
% course of the project.
\section{Technologies Used} \label{technology}
Our project combined a number of technologies to create a laser scanner.
In our initial design, we controlled an Arduino Uno in Python with the
\href{https://github.com/pyserial/pyserial}{pySerial} library; we also
captured images with two Logitech webcams. We still ended up using the
Arduino in the final design, but only because it was a convenient way to
power our laser; any 5-volt power source would have been sufficient.
When testing our initial design, we also found that the resolution of
the Logitech webcams was too low to produce good results, so we instead
designed a mount for an iPhone so that we would be able to scan with a
higher-resolution camera. All of the image processing was performed with
\href{http://opencv.org}{OpenCV}.

% 1/2 page
% Discuss what is new about your attack on the problem.
% Describe any invention or novel approaches used.
\section{Innovations}

As discussed in Section~\ref{background}, the concept of a DIY laser scanner is
not new: a number of designs already exist on websites like Instructables. The
three desirable characteristics of a DIY laser scanner are affordability,
scalability, and portability. It appears that most preexisting DIY scanners have
two out of three of these: we could not find any that had all three, so we created
them.

One unique approach that we took was building a scanner that was highly modular.
Many scanner designs on the Internet are designs for an entire scanner, meaning
that it is not easy to make simple part swaps in order to change the scale at which
the scanner can operate. Our scanner was designed specifically to come apart, so
it is very easy to, with the same printed parts, change the scale by simply swapping
out the rods. In fact, the advantage of the modularity of our design is twofold:
it automatically gives us the easy portability that we wanted as well.

% 1 page
% Describe the outcome of your efforts.
% What worked and what didn't work?
% What surprised you?
\section{Results} \label{results}

We were able to successfully design a laser scanner that met all three of our
goals: it was highly affordable, very scalable, and easy to transport.
Not counting the Arduino, which, as discussed in Section~\ref{technology},
was only used as a power source and could easily be replaced by something
simpler, or the iPhone that we used for image (actually, video) capture,
this laser scanner is highly affordable: we bought two lasers from
\href{http://www.amazon.com/Focusable-650nm-Module-driver-Plastic/dp/B00S1EXW3Y}{Amazon}
for under \$5, and the three dowels were about \$3 in total. The rest of the
parts were 3D printed, so the scanner cost under \$10 total to build, taking into
account an approximation of the cost of filament. This is more affordable than
the designs shown in both~\cite{dentroman} and~\cite{sardau}.

Also as we had planned, our laser scanner is scalable and very portable.
The only part of our design that limits the size of an object that can be scanned
is the rod length: if one wants to scan larger objects, then larger dowels can
be purchased. The entire scanner also easily comes apart into a few small parts
which can be transported and quickly put together somewhere else; again, the
limiting factor in portability is the size of the dowels.

The software was less successful than the hardware. Part of the reason for this
may be that our image processing techniques were not highly sophisticated. As
suggested by~\cite{instructions}, we first produced a binary image by thresholding
on the red value in each frame of the captured video, and then we performed the
a Hough transform to determine which lines corresponded to the projection of
the laser on the wall and the floor. However, the Hough transform had a
tendency to find erroneous lines; we did our best to fix this issue, but in
the end, it was not perfect, and the ``floor'' and ``wall'' that our software
finds are not necessarily the true floor and wall. This may have led to some
of our issues as well. \cite{instructions} describes software in which, after
the Hough transform is performed, the user is presented with multiple options
for the ``correct'' lines. Implementing this likely could have improved our results.

We were surprised by the fact that the image processing used for producing scans
was relatively simple: it essentially amounts to determining where the floor and
the wall are and then performing a number of projections.

% 1/2 page
% What did you and your team learn from this project?
% What would you do differently next time?
% What advice would you give next year's students?
\section{Lessons Learned}

This project taught us about how laser scanners work, and about the image processing
that needs to be done in order to successfully produce a 3D scan of a real-world object.

Something that we discussed while building this (but unfortunately at a time that
was too late to change course) was creating a laser scanner that would be even more
portable, by fabricating what would essentially amount to a smartphone case; the laser
could then be mounted on the phone, and the phone could combine camera data with
accelerometer and gyroscope data in order to produce a 3D scan. AutoCAD 123D Catch
is software that has the aim of producing 3D scans by using this data, but, as we
experienced during Lab 4, it has a multitude of issues. Perhaps using the laser
scanner technique, rather than combining pictures, could yield better results.
This would be an interesting approach to try in the future.

The main piece of advice that we would give next year's students would be to start
early and to work on the software and hardware at the same time if possible. Before
designing anything for our scanner, we took sample photos and videos in order to
provide an approximation of the final result. This allowed us to divide the work
and design the hardware and software in parallel, which significantly sped up the
design process.

% 1/2 page
% Summarize the overall project; what were the most important results?
% Summarize the impact of the project on the specific aims.
\section{Conclusion}

As discussed in Section~\ref{results}, this project was a moderate success.
On the hardware side, we were extremely successful: the design is simple,
the total cost of the parts is very low, it is easy to scale, and transporting
it is as easy as transporting the dowels that are used. We believe that, with
these features, our laser scanner fills an important hole in the spectrum of
existing DIY laser scanners.

Although we were not able to produce STL files or XYZ files (point clouds) that
accurately captured the objects we wished to scan, the physical design of our
laser scanner was successful and met our aims (on the hardware side).
The only shortcoming of this project was in data processing. It is unclear exactly
why this is, but the point clouds that we produced do not resemble the objects that
we scan.  The main reference that we used for implementing the software
was~\cite{instructions}, which had a tendency to be brief to a fault; it is entirely
possible that a mistranslation from this document to code was responsible for our software
issues.  Perhaps with more time, it would have been possible to produce successful scans.

\bibliographystyle{plain} \bibliography{refs}

% Summarize what each team member contributed to the project and report.
\section{Team Member Contributions}
\begin{description}
    \item[Noah] Majority of programming, report
    \item[Catherine] Design, OpenSCAD modeling, presentation
\end{description}

\appendix
% Include photos showing progress and final results.
\section{Photos}

% Include code and models written for the project.
\section{Code Listings}
All of the code for this project (including OpenSCAD code, as well as some sample
scanning photos and videos) is available on
\href{https://github.com/nhalford/3d-scanner/}{GitHub}. Our Python program is
\texttt{src/python/hough.py}.

\end{document}
