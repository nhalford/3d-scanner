\documentclass[12pt, letterpaper]{article}

\usepackage{hyperref}
\hypersetup{colorlinks=true, urlcolor=blue}

\title{CMSC 22010 Final Project: 3D Scanner}
\author{Noah Halford\\
    \href{mailto:nhalford@uchicago.edu}{nhalford@uchicago.edu}
        \and
        Catherine Moresco\\
        \href{mailto:cmoresco@uchicago.edu}{cmoresco@uchicago.edu}}
\date{June 3, 2016}
%TODO: Team name

\begin{document}

\maketitle

% 1/2 page
% Describe what you planned to accomplish with the project.
% What specifically will be created, demonstrated, invented?
% What do you expect to learn from this project?
\section{Specific Aims of the Project}

% 1 page
% Discuss previous work on this problem by others.
% Discuss sources of information or designs you are leveraging.
% Discuss shortcomings of previous efforts.
\section{Background Research}

% 2 pages
% Describe in some detail how you went about designing and building.
% Describe the steps you went through to develop ideas and prototypes.
% Describe any changes in your plan from the pre-proposal stage.
\section{Approach}

% 1/2 page
% Describe any novel technology that you used or developed during the
% course of the project.
\section{Technologies Used}
The laser scanner is controlled with an Arduino Uno, running a simple program that
rotates moves the stepper motor by one step when it receives a message. Communication
with the Arduino is controlled by a Python program which made use of the
\href{https://github.com/pyserial/pyserial}{pySerial} library in order to send messages
to the Arduino. Data is captured by two Logitech webcams, which are controlled with
\href{http://opencv.org}{OpenCV}.

% 1/2 page
% Discuss what is new about your attack on the problem.
% Describe any invention or novel approaches used.
\section{Innovations}

% 1 page
% Describe the outcome of your efforts.
% What worked and what didn't work?
% What surprised you?
\section{Results}
One of the challenges we faced in this project was designing the outer section (that is,
the camera and laser mounts). We decided that the inner radius would be 65 mm so that
we would be able to scan reasonably large objects. However, this meant that the outer
section of our scanner was too large to be printed in one piece. Because we needed
the cameras and laser to be placed in precise positions relative to each other in order
for our software to work, it was necessary for us to print the mounts in pieces that
could connect to each other and lock. This was first attempted by creating tabs in
the pieces that could slide into each other, but this proved difficult to print because
the necessary overhangs sagged, and support was nearly impossible to remove because of
how close to the bed the tabs were printed. The solution we found was to instead use
puzzle piece-like connections, but this came with difficulties as well: PLA changes size
slightly when it dries, so digitally fabricated objects which fit together perfectly
did not actually fit together after being printed.

% 1/2 page
% What did you and your team learn from this project?
% What would you do differently next time?
% What advice would you give next year's students?
\section{Lessons Learned}

% 1/2 page
% Summarize the overall project; what were the most important results?
% Summarize the impact of the project on the specific aims.
\section{Conclusion}

% TODO: Citations/references

% Summarize what each team member contributed to the project and report.
\section{Team Member Contributions}

\appendix
% Include photos showing progress and final results.
\section{Photos}

% Include code and models written for the project.
\section{Code Listings}

\end{document}
