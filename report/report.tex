\documentclass[12pt, letterpaper]{article}

\usepackage{hyperref}
\hypersetup{colorlinks=true, urlcolor=blue}

\title{CMSC 22010 Final Project: 3D Scanner}
\author{Noah Halford\\
    \href{mailto:nhalford@uchicago.edu}{nhalford@uchicago.edu}
        \and
        Catherine Moresco\\
        \href{mailto:cmoresco@uchicago.edu}{cmoresco@uchicago.edu}}
        \date{{\Large Team Scanner Planner}\\[5mm] June 3, 2016}

\begin{document}

\maketitle

% 1/2 page
% Describe what you planned to accomplish with the project.
% What specifically will be created, demonstrated, invented?
% What do you expect to learn from this project?
\section{Specific Aims of the Project} \label{aims}

One possible use case for 3D printers is object replication. While it certainly is possible
to design 3D models of real-world objects in CAD software, there are a number of drawbacks
to this approach: it may require many iterations to get right, it may be time-consuming, and
it requires a very particular set of skills. One alternative is to use a 3D scanner, which
simplifies and automates the process of digitally fabricating the desired object.

The goal of this project was to design and build a 3D scanner that could ultimately be used
to produce STL file approximations of real-world objects. Of course, because we were doing
this cheaply, we did not anticipate perfect results: our attempt to create a laser scanner
would be considered successful if we were able to produce STL files that were recognizable
as the objects they represented. We did not expect that our resulting STLs would necessarily
be completely faithful representations of the original objects.


% 1 page
% Discuss previous work on this problem by others.
% Discuss sources of information or designs you are leveraging.
% Discuss shortcomings of previous efforts.
\section{Background Research}

% TODO

Laser scanning is a technology that has already been very well-developed; going into
this project, we knew that it was unlikely for us to make significant improvements
over existing designs. Our goals were to design a laser scanner that would be cheap
and would work (as described in Section~\ref{aims}), but, perhaps most importantly,
to learn about how laser scanners work. Instructions for a number of cheap DIY laser
scanners already exist on the Internet, and we used these as inspiration for our
project---although we did design our own hardware and software.
% TODO: Cite Instructables.

% 2 pages
% Describe in some detail how you went about designing and building.
% Describe the steps you went through to develop ideas and prototypes.
% Describe any changes in your plan from the pre-proposal stage.
\section{Approach}

One of the challenges that we faced in this project was designing the laser scanner.
We went through a number of design iterations, and the final product changed substantially
from our plan from the pre-proposal stage. Our original plan was to have two
webcams and the lasers between them mounted on a circular arc, with a turntable that
could turn independently. The object to be scanned would sit on the turntable, which
would rotate with a stepper motor. At each step of the motor, each camera would capture
an image; we hoped to be able to combine data about rotation with a number of distance
measurements (the distance between the cameras, between the cameras and the lasers, etc.)
in order to produce a 3D model representation of the object we wished to scan.
We built a prototype of this kind of scanner, but ultimately decided that it would % TODO: Pictures?
too difficult to appropriately deal with the data that we produced. Our prototype used
an Arduino attached to the stepper motor to control the turntable. A Python program
communicating with the Arduino allowed us rotate the stepper motor (that is, the turntable)
one step at a time, and take a photo from each webcam between steps. This produced
1024 images (512 per camera), but because of the low resolution of the webcams and
the lack of precision in our constructed laser scanner, we decided that this approach
would not work.

We finally settled on a different, arguably simpler design.

% 1/2 page
% Describe any novel technology that you used or developed during the
% course of the project.
\section{Technologies Used}
%The laser scanner is controlled with an Arduino Uno, running a simple program that
%rotates moves the stepper motor by one step when it receives a message. Communication
%with the Arduino is controlled by a Python program which made use of the
%\href{https://github.com/pyserial/pyserial}{pySerial} library in order to send messages
%to the Arduino. Data is captured by two Logitech webcams, which are controlled with
%\href{http://opencv.org}{OpenCV}.

% 1/2 page
% Discuss what is new about your attack on the problem.
% Describe any invention or novel approaches used.
\section{Innovations}

% 1 page
% Describe the outcome of your efforts.
% What worked and what didn't work?
% What surprised you?
\section{Results}

% 1/2 page
% What did you and your team learn from this project?
% What would you do differently next time?
% What advice would you give next year's students?
\section{Lessons Learned}

% 1/2 page
% Summarize the overall project; what were the most important results?
% Summarize the impact of the project on the specific aims.
\section{Conclusion}

% TODO: Citations/references

% Summarize what each team member contributed to the project and report.
\section{Team Member Contributions}

\appendix
% Include photos showing progress and final results.
\section{Photos}

% Include code and models written for the project.
\section{Code Listings}

\end{document}
