\documentclass[12pt, letterpaper]{article}

\usepackage{hyperref}
\hypersetup{colorlinks=true, urlcolor=blue}

\title{CMSC 22010 Final Project Proposal: 3D Scanner}
\author{Noah Halford and Catherine Moresco}
\date{April 29, 2016}

\begin{document}

\maketitle

\section{Objective}

We aim to construct a simple laser scanner which will have the ability to scan
real-world objects and produce STL representations of them, ready to be 3D-printed.

Thus, our project will have two main components: the first will be the hardware---that is,
the scanner itself---which will be constructed from a combination of electronic parts
and parts that we will design and 3D print. The second component is the software, which
will need to:
\begin{enumerate}
    \item Control the 3D scanner
    \item Interpret the data produced by the scanner and use it to produce an STL file
\end{enumerate}

\section{Background and Previous Work}

One possible use case for 3D printers is object replication. While it certainly is possible
to design 3D models of real-world objects in CAD software, there are a number of drawbacks
to this approach: it may require many iterations to get right, it may be time-consuming, and
it requires a very particular set of skills. One alternative is to use a 3D scanner, which
simplifies and automates the process of digitally fabricating the desired object.

3D scanners are already in wide use (see Section~\ref{refs}) and commercially
available.\footnote{Amazon lists 39 available for purchase, ranging widely in price from under
\$200 to nearly \$50,000.} Moreover, there are already a number of DIY solutions readily
available on the Internet (some are listed in Section~\ref{refs}), rendering the cost of
commercially-produced 3D scanners largely irrelevant. As such, the aims of this project
will be far more pedagogical than practical.

\section{Technical Approach}

Section~\ref{refs} lists a number of references that will be useful in designing a
laser scanner. The most straightforward approach appears to be a turntable on which
to put the object to scan, which is able to rotate independently of the laser itself.
Parts for the 3D scanner which cannot be cannibalized from our 3D printers will be
designed in CAD software and 3D printed.

The biggest challenge will likely be in processing the data produced by the scanner.
This can likely be done by combining information about motor positions with the
other output data.

\section{References} \label{refs}

Below are some DIY guides that might serve as inspiration in our design:
\begin{itemize}
    \item \url{http://www.instructables.com/id/3-D-Laser-Scanner/}
    \item \url{http://www.instructables.com/id/Build-a-30-laser/}
\end{itemize}
Other reference:
\begin{itemize}
    \item \url{https://en.wikipedia.org/wiki/3D\_scanner}
    \item \url{http://warnercnr.colostate.edu/~lefsky/isprs/1133.pdf}
    \item \url{http://www.rockini.name/research/papers/vcgscanner.pdf}
    \item \url{http://www.mdpi.com/2227-7080/2/2/76}
\end{itemize}

\section{Timeline}

\begin{description}
    \item[May 8] Design
    \item[May 15] Construction completed
    \item[May 22] Transfer of scanner data to computer
    \item[May 29] Conversion of scanner data to STL
\end{description}

\end{document}
